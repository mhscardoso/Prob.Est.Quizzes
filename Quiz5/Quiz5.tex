\documentclass[12pt]{report}

\usepackage[utf8]{inputenc}
\usepackage[brazilian]{babel}
\usepackage{geometry}
\usepackage{amsmath}
\usepackage{amsfonts}
\usepackage{tikz, pgfplots}
\usepackage{setspace}
\usepackage{multirow}

\newgeometry{
	margin=0.5in
}

\setstretch{1.5}
\setlength{\parindent}{0pt}
\pgfplotsset{compat=1.18}

\newenvironment{boldenv}
  {\bfseries}% \begin{boldenv}
  {}% \end{boldenv}

\title{Estatística e Modelos Probabilísticos - Quiz 4}
\author{
	Matheus Henrique Sant'Anna Cardoso\\
	DRE: 121073530
}
\date{Novembro de 2022}

\begin{document}
    \maketitle \newpage

%% Questão 1 -----------------------------------------------------------------------------------------------
\begin{boldenv}
(1) Após um bom tempo consegui juntar com o rendimento da poupança ($0.6\%$) um total de $P = 10000$ reais. Pretendo trocar para aplicações cujas taxas de retorno são variáveis aleatórias independentes $x_1 e x_2$, com médias 5\% e 14\% e desvios padrão 1\% e 8\%, respectivamente. Para avaliar o risco da aplicação utilizarei o desvio padrão, $\sigma(R)$, do seu retorno total $R=P_1x_1 + P_2x_2$.
\end{boldenv}

\begin{boldenv}
(a) para ter o risco mínimo possível, que quantias $P_1$ e $P_2$ devo investir nas respectivas aplicações? Quais são a média do retorno e o risco correspondentes?
\end{boldenv}

\begin{boldenv}
(b) qual é o tamanho do risco que devo correr para obter o mesmo rendimento em reais da poupança? Vale a pena trocar de investimento?
\end{boldenv}

\begin{boldenv}
(c) Através da Desigualdade de Chebyshev, obtenha um intervalo simétrico em torno de 770 reais que, com probabilidade superior a 80\%, conterá o retorno $R$ da carteira obtida no item (b).
\end{boldenv}


\vspace{\baselineskip}

%% Questão 2 ----------------------------------------------------------------------------------------------
\begin{boldenv}
(2) Na resolução do exercício abaixo foi cometido um erro grave. Pergunta: Sejam $x$ e $y$ duas v.a.'s independentes e tais que $x \sim N (80; 9)$ e $y \sim N (50; 16)$. Qual a distribuição de probabilidade da v.a. $z = x - y$ ?

Resposta:
\end{boldenv}
\begin{align*}
E[z] = E[x - y] = E[x] - E[y] = 80 - 50 = 30\\
Var(z) = Var(x - y) = Var(x) - Var(y) = 9 - 16 = -7\\
\text{Conclusão}: z \sim N (30; -7)
\end{align*}

\begin{boldenv}
(a) Qual foi o erro cometido aqui?
\end{boldenv}

\begin{boldenv}
(b) qual a solução correta
\end{boldenv}

\vspace{\baselineskip}

%% Questão 3 ---------------------------------------------------------------------------------------------
\begin{boldenv}
(3) Considere duas variáveis aleatórias $x$ e $y$ com função densidade de probabilidade conjunta
\end{boldenv}

\begin{align*}
    f_{X, Y}(x, y) =
    \begin{cases}
        kxy^2 & \text{se } 0 \leq x \leq 1 \text{ e } 0 \leq y \leq 2\text{,}\\
        0     & \text{caso contrário, }
    \end{cases}
\end{align*}
\begin{boldenv}
Onde $K$ é uma constante real. Determine $E[X],\, E[Y],\, E[X \mid Y = 1] \text{ e } E[Y \mid X = 0]$.
\end{boldenv}

\vspace{\baselineskip}

%% Questão 4 ---------------------------------------------------------------------------------------------
\begin{boldenv}
(4) A chegada de passageiros a uma paragem de autocarro segue um processo de Poisson com intensidade $\lambda_1$. Seja $T$ o tempo de chegada de um autocarro que é independente do processo de Poisson. Quando $t = 0$ não existem passageiros na paragem. Supondo que $T$ segue uma distribuição exponencial com intensidade $\lambda_2$, calcule o número médio de pessoas na paragem no instante $T$.

Resposta: $\frac{\lambda_1}{\lambda_2}$
\end{boldenv}

\vspace{\baselineskip}

%% Questão 5 ---------------------------------------------------------------------------------------------

\begin{boldenv}
(5) Uma concessionária de automóveis vem mantendo semanalmente em estoque 2 carros importados e 3 de fabricação nacional, para atender aos seus clientes. Sejam $x$ e $y$ as variáveis aleatórias que representam respectivamente o número de carros importados e o número de carros nacionais que ela vende ao longo de uma semana. Assim sendo, $x$ pode assumir os valores $0,\,1,\,2$ e $y$ os valores $0,\,1,\,2,\,3$. A função de probabilidade conjunta de $x$ e $y$ é dada pela tabela abaixo:
\end{boldenv}
\begin{center}
    \begin{tabular}{ | c | c | c | c | c | }
        \hline
        \multirow{2}{*}{$x$} & \multicolumn{4}{| c |}{$y$}\\ \cline{2-5}
          & 0 & 1 & 2 & 3\\
        \hline
        0 & 0.01 & 0.05 & 0.05 & 0.04\\
        \hline
        1 & 0.05 & 0.20 & 0.15 & 0.10\\
        \hline
        2 & 0.04 & 0.15 & 0.10 & 0.06\\
        \hline
    \end{tabular}
\end{center}

\begin{boldenv}
    Qual a probabilidade de que, em uma determinada semana:
\end{boldenv}

\begin{boldenv}
    (a) não seja vendido nenhum carro importado?
\end{boldenv}

\begin{boldenv}
    (b) todos os carros nacionais sejam vendidos?
\end{boldenv}

\begin{boldenv}
    (c) sejam vendidos no máximo um carro importado e um carro nacional?
\end{boldenv}

\begin{boldenv}
    (d) sejam vendidos mais carros importados do que nacionais?
\end{boldenv}

\begin{boldenv}
    (e) sejam vendidos ao todo pelo menos 4 carros?
\end{boldenv}

\begin{boldenv}
    Obtenha:
\end{boldenv}

\begin{boldenv}
    (f) as distribuições marginais de $x$ e de $y$.
\end{boldenv}

\begin{boldenv}
    (g) as distribuições condicionais de $x$ dado $y$, e de $y$ dado $x$.
\end{boldenv}

\begin{boldenv}
    (h) $Cov(x, y)$ e $\rho(x, y)$
\end{boldenv}

\vspace{\baselineskip}

%% Questão 6 ---------------------------------------------------------------------------------------------

\begin{boldenv}
    (6) O tempo de vida de um equipamento é uma variável aleatória $x$ com função densidade de probabilidade exponencial de parâmetro $a$, ou seja,
\end{boldenv}
\[p_x(X) = ae^{-aX}u(X).\]

\begin{boldenv}
    Quando o equipamento falha, o tempo necessário para repará-lo pode ser modelado por uma variável aleatória $y$, com função densidade de probabilidade dada por
\end{boldenv}
\[p_y(Y) = be^{-bY}u(Y).\]

\begin{boldenv}
    Seja $z$ a duração total de um ciclo do equipamento, entendendo-se como tal o tempo decorrido desde que o equipamento entra em operação até que ele volte a funcionar depois de ter falhado uma vez, ou seja,
\end{boldenv}
\[z = x + y\]

\begin{boldenv}
    (a) Determine o coeficiente de correlação $\rho_{xz}$ entre as variáveis aleatórias $x$ e $z$.
\end{boldenv}

\begin{boldenv}
    (b) Determine a função densidade de probabilidades $\rho_z(Z)$ da variável aleatória $z$. Para isso, lembre-se que
\end{boldenv}
\[\frac{1}{(a - jv)(b - jv)} = \frac{1}{b - a} \left(\frac{1}{a - jv} - \frac{1}{b - jv}\right)\]

\begin{boldenv}
    (c) Calcule, para $a = 1$ e $b = 5$, a probabilidade de que a duração do ciclo exceda 10.
\end{boldenv}

\begin{boldenv}
    (d) Determine um limitante superior para a probabilidade calculada no item anterior, utilizando a desigualdade de Tchebyshev.
\end{boldenv}

\vspace{\baselineskip}

%% Questão 7 ---------------------------------------------------------------------------------------------

\begin{boldenv}
    (7) Num determinado instante de $t_0$ uma chamada telefônica chega a uma central
    telefônica e encontra todos os seus $N$ circuitos ocupados. Assim, a chamada não pode
    ser servida imediatamente e passa a aguardar a liberação de qualquer um dois $N$
    circuitos. Considere que as durações $x_i$, $i = 1, \cdots , N$ das chamadas que estão utilizando
    cada um dos $N$ circuitos (medidas a partir de $t_0$ ) são variáveis aleatórias estatisticamente
    independentes e identicamente distribuídas, com funções de densidade de
    probabilidade dadas por
\end{boldenv}
\[p_{x_i}(X) = e^{-X}u(X); \,\,\,\, i = 1, 2, \cdots, N\]

\begin{boldenv}
    Determine a espera média da chamada telefônica, ou seja, o intervalo de tempo
ocorrido desde $t_0$ até o instante em que a chamadas é servida.
\end{boldenv}

\end{document}