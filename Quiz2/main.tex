\documentclass[11pt,a4paper]{article}
\usepackage{setspace}
\usepackage[utf8]{inputenc}
\usepackage[portuguese]{babel}
\usepackage{amsmath}
\usepackage{amssymb}
\usepackage[left=2cm,right=2cm,top=2cm,bottom=2cm]{geometry}
\usepackage{pgfplots}
\setstretch{1.25}
\setlength{\parindent}{0pt}
\pgfplotsset{width=10cm,compat=1.9}

\author{
	Matheus Henrique SantAnna Cardoso\\
	DRE: 121073530
}
\title{Quiz 2 - Estatística e Métodos Probabilísticos}
\date{Setembro de 2022}

\begin{document}

\maketitle

{\large{\textbf{Questão 1}}}\\
\textbf{(a)} Vamos iniciar pensando no espaço amostra dos dois casos:

O $\Omega$ nesse caso será todas as possibilidades de soma de duas faces de dois dados, logo:
\[\Omega = \{2, 3, 4, 5, 6, 7, 8, 9, 10, 11, 12\}\]
que são todas as possibilidades de soma.

A $\sigma$-álgebra mínima para a classe \{A, B\} será aquela que contempla os dois casos, devendo ter um caso com a união de todos os casos cuja soma é um número primo (A) e um subconjunto com o valor 7 (B).

O que deve ter no conjunto: $\emptyset, \Omega, \{7\},\{2, 3, 5, 7, 11\}, \{2, 3, 5, 11\}$. Devemos adicionar, portanto, os complementares dos dois últimos conjuntos.

\{$\emptyset, \Omega$, \{7\}, \{2, 3, 4, 5, 6, 8, 9, 10, 11, 12\}, \{2, 3, 5, 7, 11\}, \{4, 6, 8, 9, 10, 12\}\}, \{2, 3, 5, 11\}, \{4, 6, 7, 8, 9, 10, 12\}\}

\vspace{\baselineskip}

\noindent \textbf{(b)} Os casos mostrados no $\Omega$ e na $\sigma$-álgebra não são equiprováveis. Veja que existem 36 ($6 \cdot 6$) possibilidades para a queda dos dados.

Para o caso A, devemos calcular quantas possibilidades existem para cada soma cujo resultado é primo. Seja $S_{i}$ o número de possibilidades de cair a soma $i$.
\begin{align*}
	S_{2}  &= 1\\
	S_{3}  &= 2\\
	S_{5}  &= 4\\
	S_{7}  &= 6\\
	S_{11} &= 2
\end{align*}
Podemos, então, dizer que a soma de todos os casos primos será $S_{primo} = 1 + 2 + 4 + 6 + 2 = 15$.
Dessa forma, a probabilidade da soma ser ímpar é $P(A) = \frac{15}{36} = \frac{5}{12}$.

Já a probabilidade $P(B)$ de a soma ser 7 é $\frac{6}{36}$, sendo todas as possibilidades de a soma ser 7 dividido por todos os casos. Assim $P(B) = \frac{1}{6}$.

Agora, devemos calcular $P(A/B)$, que é a probalilidade de a soma ser primo dado que vale 7. Ora, se já sabemos que vale 7, a soma é um número primo. Portanto, $P(A/B) = 1$, pois é certo que será primo.

Finalmente, devemos calcular $P(B/A)$, que é a probalilidade de ser 7 dado que é primo. Nisso, podemos utilizar o resultado anterior para fazer:
\begin{align*}
	P(B/A) &= \frac{P(B \cap A)}{P(A)}\\
	P(A/B) &= \frac{P(B \cap A)}{P(B)} = 1 \,\,\text{do último ítem}\\
	P(B \cap A) &= P(B)\\
	P(B/A) &= \frac{P(B)}{P(A)}\\
	P(B/A) &= \frac{1 / 6}{5 / 12} = \frac{2}{5}
\end{align*}
Assim, chegamos ao resultado que $P(B/A) = \frac{2}{5}$

\vspace{\baselineskip}

\noindent \textbf{(c)} Para sabermos se não são mutuamente exclusivos, basta termos um evento que é comum aos dois. Nisso, é fácil ver que quando ocorre o evento cuja soma é 7, ambos os casos (A e B) são satisfeitos, pois é primo (A) e vale 7 (B). Dessa forma, nota-se que os eventos não são mutuamente exclusivos.

Além disso, pensando logicamente, percebemos que os eventos não são independentes. Perceba que se o número não for primo, necessariamente não poderá ser 7. Ao contrário, caso a soma seja 7, necessariamente será primo. Matematicamente, fazemos:
\[P(A) = \frac{5}{12}\]
\[P(B) = \frac{1}{6}\]
\[P(A \cap B) = \frac{1}{6}\]
Veja, finalmente, que $P(A) \cdot P(B) \neq P(A \cap B)$. Mostrando que os eventos não são independentes.

\newpage

{\large{\textbf{Questão 2}}}

\newpage

{\large{\textbf{Questão 3}}}

\vspace{\baselineskip}

\begin{tikzpicture}
\draw[->] (-5.5, 0) -- (5.5, 0) node[below] {$V$};
\draw (-5, -0.1) -- (-5, 0.1) node[below] {$-5$};
\draw (-4, -0.1) -- (-4, 0.1) node[below] {$-4$};
\draw (-3, -0.1) -- (-3, 0.1) node[below] {$-3$};
\draw (-2, -0.1) -- (-2, 0.1) node[below] {$-2$};
\draw (-1, -0.1) -- (-1, 0.1) node[below] {$-1$};
\draw (0, -0.1) -- (0, 0.1) node[below] {$\ 0$};
\draw (1, -0.1) -- (1, 0.1) node[below] {$\ 1$};
\draw (2, -0.1) -- (2, 0.1) node[below] {$\ 2$};
\draw (3, -0.1) -- (3, 0.1) node[below] {$\ 3$};
\draw (4, -0.1) -- (4, 0.1) node[below] {$\ 4$};
\draw (5, -0.1) -- (5, 0.1) node[below] {$\ 5$};
\end{tikzpicture}

\vspace{\baselineskip}
Acima, o espaço amostra do problema, que é o intervalo $[-5, 5]$. Com isso em mente, sabemos que para um erro de $0.2$, é preciso que a voltagem esteja em u intervalo do tipo $X \pm 0.2$, sendo $X$ um número inteiro entre $-4$ e $4$. Para os casos em que $X = \pm 5$, a variação será de $[-5, -4.8]$ (para o caso negativo) e $[4.8, 5]$ (para o caso positivo).

No primeiro caso, temos nove casos com intervalos de $0.4$. No segundo, dois casos com um intervalo de $0.2$. Ao final, temos:

Intervalos válidos: $9 \cdot 0.4 + 2 \cdot 0.2 = 3.6 + 0.4 = 4$.

O tamanho total do intervalo é: $5 - (-5) = 10$.

A probabilidade é de $\frac{4}{10} = 0.4$. Sendo de 40\% a probabilidade de que seja apontado um erro inferior à $0.2$.

\newpage

{\large{\textbf{Questão 4}}}

(a) Aqui, podemos utilizar dois eventos. O primeiro, conta o número de homens dentre os filhos de um casal aleatório (com $\Omega_{1}$ sendo  o espaço amostra), o segundo, o número de mulheres ($\Omega_{2}$).

Temos que:

\[\Omega_{1} = \Omega_{2} = \{0, 1, 2\}\]

com o mesmo $\sigma$-álgebra, inclusive.

\vspace{\baselineskip}

\begin{tikzpicture} 
\begin{axis}
% Primeira fileira x
\addplot3 coordinates 
{
	(0,0,0)
	(0, 0, 0.33)
};

\addplot3 coordinates 
{
	(1,0,0)
	(1, 0, 0.16)
};

\addplot3 coordinates 
{
	(2,0,0)
	(2, 0, 0.083)
};

% Segunda fileira
\addplot3 coordinates 
{
	(0,1,0)
	(0, 1, 0.16)
};

\addplot3 coordinates 
{
	(1,1,0)
	(1, 1, 0.16)
};

\addplot3 coordinates 
{
	(2,1,0)
};

% Terceira fileira
\addplot3 coordinates 
{
	(0,2,0)
	(0, 2, 0.083)
};

\addplot3 coordinates 
{
	(1,2,0)
};

\addplot3 coordinates 
{
	(2,2,0)
};
\end{axis}
\end{tikzpicture}

\vspace{\baselineskip}
\vspace{\baselineskip}

(b) Nota-se que os eventos não são independentes. De início, podemos pensar que caso um casal tenha dois filhos homens, por exemplo, não poderá ter filhas mulheres, ou seja, no final, um evento influencia o outro.

\end{document}
