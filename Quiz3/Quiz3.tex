\documentclass[a5paper]{report}

\usepackage[utf8]{inputenc}
\usepackage[brazilian]{babel}
\usepackage{geometry}
\usepackage{amsmath}
\usepackage{amsfonts}
\usepackage{tikz, pgfplots}
\usepackage{setspace}

\newgeometry{
	margin=0.5in
}

\setstretch{1.5}
\setlength{\parindent}{0pt}
\pgfplotsset{compat=1.18}

\title{Estatística e Modelos Probabilísticos - Quiz 3}
\author{
	Matheus Henrique Sant'Anna Cardoso\\
	DRE: 121073530
}
\date{Novembro de 2022}


\begin{document}
	
\maketitle \newpage

\section*{Questão 1}
Considere o lançamento de dois dados e a experiência cujo resultado consiste na soma do número de pontos da face superior dos dados. Resolva:

\textbf{a.}
Modele com uma v.a. esta soma.

Aqui, o espaço amostra da variável aleatória pode ser dado por:

\[\Omega_x = \left\{2, 3, 4, 5, 6, 7, 8, 9, 10, 11, 12\right\}\]

De forma que, possamos, de cada par $f_i\,f_j$ com $i, j = 1, 2, \cdots, 6$ apontar para uma soma $i + j$.

\begin{center}
	\begin{tikzpicture}
		\draw (0, 0) ellipse (1 and 2);
		\draw node at (0, 2.2) {$\Omega$};
	
		\draw node (1) at (0, 1.6) {$f_1\,f_1$};
		\draw node (2) at (0, 0.8) {$f_1\,f_2$};
		\draw node (3) at (0, 0.0) {$f_1\,f_3$};
		\draw node (4) at (0, -0.8) {$\vdots$};
		\draw node (5) at (0, -1.6) {$f_6\,f_6$};
	
		\draw (5, 0) ellipse (1 and 2);
		\draw node at (5, 2.2) {$\Omega_x$};
	
		\draw node (a) at (5, 1.6) {$2$};
		\draw node (b) at (5, 0.8) {$3$};
		\draw node (c) at (5, 0.0) {$4$};
		\draw node (d) at (5, -0.8) {$\vdots$};
		\draw node (e) at (5, -1.6) {$12$};
	
		\draw[->] (1.east) to node[right] {} (a.west);
		\draw[->] (2.east) to node[right] {} (b.west);
		\draw[->] (3.east) to node[right] {} (c.west);
		\draw[->] (5.east) to node[right] {} (e.west);
		
	\end{tikzpicture}
\end{center}


\textbf{b.}
Encontre e esboce sua distribuição de probabilidade.

\begin{center}
	\begin{tikzpicture}
		\draw[->, thick] (0, 0) -- (0, 3.5);
		\draw[->, thick] (0, 0) -- (9, 0);

		\draw node at (0.6, 3.5) {$P_x(X)$};
		\draw node at (9, 0.3) {$x$};
		
		\draw node at (0.75, -0.4) {$2$};
		\draw[red, very thick] (0.75, 0) -- (0.75, 0.5);
		\draw[dashed] (0.75, 0.5) -- (0, 0.5);

		\draw node at (1.5, -0.4) {$3$};
		\draw[red, very thick] (1.5, 0) -- (1.5, 1);
		\draw[dashed] (1.5, 1) -- (0, 1);

		\draw node at (2.25, -0.4) {$4$};
		\draw[red, very thick] (2.25, 0) -- (2.25, 1.5);
		\draw[dashed] (2.25, 1.5) -- (0, 1.5);

		\draw node at (3, -0.4) {$5$};
		\draw[red, very thick] (3, 0) -- (3, 2);
		\draw[dashed] (3, 2) -- (0, 2);

		\draw node at (3.75, -0.4) {$6$};
		\draw[red, very thick] (3.75, 0) -- (3.75, 2.5);
		\draw[dashed] (3.75, 2.5) -- (0, 2.5);

		\draw node at (4.5, -0.4) {$7$};
		\draw[red, very thick] (4.5, 0) -- (4.5, 3);
		\draw[dashed] (4.5, 3) -- (0, 3);

		\draw node at (5.25, -0.4) {$8$};
		\draw[red, very thick] (5.25, 0) -- (5.25, 2.5);
		\draw[dashed] (5.25, 2.5) -- (3.75, 2.5);

		\draw node at (6, -0.4) {$9$};
		\draw[red, very thick] (6, 0) -- (6, 2);
		\draw[dashed] (6, 2) -- (3, 2);

		\draw node at (6.75, -0.4) {$10$};
		\draw[red, very thick] (6.75, 0) -- (6.75, 1.5);
		\draw[dashed] (6.75, 1.5) -- (2.25, 1.5);

		\draw node at (7.50, -0.4) {$11$};
		\draw[red, very thick] (7.50, 0) -- (7.50, 1);
		\draw[dashed] (7.5, 1) -- (1.5, 1);

		\draw node at (8.25, -0.4) {$12$};
		\draw[red, very thick] (8.25, 0) -- (8.25, 0.5);
		\draw[dashed] (8.25, 0.5) -- (0.75, 0.5);

		\draw node at (-0.4, 0.5) {$1/36$};
		\draw node at (-0.4, 1.0) {$2/36$};
		\draw node at (-0.4, 1.5) {$3/36$};
		\draw node at (-0.4, 2.0) {$4/36$};
		\draw node at (-0.4, 2.5) {$5/36$};
		\draw node at (-0.4, 3.0) {$6/36$};

	\end{tikzpicture}
\end{center}

\textbf{c.}
Encontre e esboce a Função Distribuição de Probabilidades (F.D.P.)

Sabemos que:

\[F_x(X) = \sum_i P(x = X_i)u_{-1}(X - X_i)\]

Então,
\[F_x(X) = P(2)u_{-1}(X - 2) + P(3)u_{-1}(X - 3) + \cdots P(12)u_{-1}(X - 12) =\]
\[F_x(X) = \frac{1}{36}u_{-1}(X - 2) + \frac{2}{36}u_{-1}(X - 3) + \cdots + \frac{1}{36}u_{-1}(X - 12)\]

Esboçando a Função Distribuição de Probabilidades, temos:

\begin{center}
	\begin{tikzpicture}
		
		\draw[->, thick] (0, 0) -- (0, 7.5);
		\draw[->, thick] (0, 0) -- (11, 0);

		\draw node at (0, -0.4) {$0$};
		\draw node at (0.75, -0.4) {$1$};
		\draw node at (1.5, -0.4) {$2$};
		\draw node at (2.25, -0.4) {$3$};
		\draw node at (3, -0.4) {$4$};
		\draw node at (3.75, -0.4) {$5$};
		\draw node at (4.5, -0.4) {$6$};
		\draw node at (5.25, -0.4) {$7$};
		\draw node at (6, -0.4) {$8$};
		\draw node at (6.75, -0.4) {$9$};
		\draw node at (7.5, -0.4) {$10$};
		\draw node at (8.25, -0.4) {$11$};
		\draw node at (9, -0.4) {$12$};
		\draw node at (9.75, -0.4) {$13$};

		\draw node at (0.6, 7.5) {$F_x(X)$};
		\draw node at (11, -0.4) {$X$};

		\draw[red, very thick] (0, 0) -- (1.5, 0);
		\draw[red, very thick] (1.5, 0.2) -- (2.25, 0.2);
		\draw[red, very thick] (2.25, 0.6) -- (3, 0.6);
		\draw[red, very thick] (3, 1.2) -- (3.75, 1.2);
		\draw[red, very thick] (3.75, 2) -- (4.5, 2);
		\draw[red, very thick] (4.5, 3) -- (5.25, 3);
		\draw[red, very thick] (5.25, 4.2) -- (6, 4.2);
		\draw[red, very thick] (6, 5.2) -- (6.75, 5.2);
		\draw[red, very thick] (6.75, 6) -- (7.5, 6);
		\draw[red, very thick] (7.5, 6.6) -- (8.25, 6.6);
		\draw[red, very thick] (8.25, 7) -- (9, 7);
		\draw[red, very thick] (9, 7.2) -- (9.75, 7.2);
		\draw[red, very thick] (9.75, 7.2) -- (10.5, 7.2);

		\draw[dashed] (1.5, 0) -- (1.5, 0.2);
		\draw[dashed] (2.25, 0) -- (2.25, 0.6);
		\draw[dashed] (3, 0) -- (3, 1.2);
		\draw[dashed] (3.75, 0) -- (3.75, 2);
		\draw[dashed] (4.5, 0) -- (4.5, 3);
		\draw[dashed] (5.25, 0) -- (5.25, 4.2);
		\draw[dashed] (6, 0) -- (6, 5.2);
		\draw[dashed] (6.75, 0) -- (6.75, 6);
		\draw[dashed] (7.5, 0) -- (7.5, 6.6);
		\draw[dashed] (8.25, 0) -- (8.25, 7);
		\draw[dashed] (9, 0) -- (9, 7.2);
		\draw[dashed] (10.5, 0) -- (10.5, 7.2);

		\draw node at (-0.5, 0.1) {$\frac{1}{36}$};
		\draw node at (-0.5, 0.6) {$\frac{3}{36}$};
		\draw node at (-0.5, 1.2) {$\frac{6}{36}$};
		\draw node at (-0.5, 2) {$\frac{10}{36}$};
		\draw node at (-0.5, 3) {$\frac{15}{36}$};
		\draw node at (-0.5, 4.2) {$\frac{21}{36}$};
		\draw node at (-0.5, 5.2) {$\frac{26}{36}$};
		\draw node at (-0.5, 6) {$\frac{30}{36}$};
		\draw node at (-0.5, 6.6) {$\frac{33}{36}$};
		\draw node at (0.5, 7) {$\frac{35}{36}$};
		\draw node at (-0.5, 7.2) {$1$};

	\end{tikzpicture}
\end{center}

\textbf{d.}
Encontre e esboce a função densidade de probabilidades (f.d.p.)

Sabemos que $\displaystyle p_x(X) = \frac{dF_x(X)}{dX}$. Dessa forma, vamos derivar a F.D.P. para obtermos f.d.p.

\[p_x(X) =
\frac{1}{36} \frac{du(X - 2)}{dX} +
\frac{2}{36} \frac{du(X - 3)}{dX} +
\frac{3}{36} \frac{du(X - 4)}{dX} +
\cdots +
\frac{1}{36} \frac{du(X - 12)}{dX}
\]

Assim:
\[
p_x(X) =
\frac{1}{36} \delta(X - 2) +
\frac{2}{36} \delta(X - 3) +
\frac{3}{36} \delta(X - 4) +
\cdots +
\frac{1}{36} \delta(X - 12)
\]

Esboçando, teremos:

\begin{center}
	\begin{tikzpicture}
		
		\draw[->, thick] (0, 0) -- (0, 6.5);
		\draw[->, thick] (0, 0) -- (10, 0);

		\draw node at (0, -0.4) {$0$};
		\draw node at (0.75, -0.4) {$1$};
		\draw node at (1.5, -0.4) {$2$};
		\draw node at (2.25, -0.4) {$3$};
		\draw node at (3, -0.4) {$4$};
		\draw node at (3.75, -0.4) {$5$};
		\draw node at (4.5, -0.4) {$6$};
		\draw node at (5.25, -0.4) {$7$};
		\draw node at (6, -0.4) {$8$};
		\draw node at (6.75, -0.4) {$9$};
		\draw node at (7.5, -0.4) {$10$};
		\draw node at (8.25, -0.4) {$11$};
		\draw node at (9, -0.4) {$12$};

		\draw node at (0.6, 6.8) {$p_x(X)$};
		\draw node at (10, -0.4) {$X$};

		\draw[->, red, very thick] (1.5, 0) -- (1.5, 1);
		\draw[->, red, very thick] (2.25, 0) -- (2.25, 2);
		\draw[->, red, very thick] (3, 0) -- (3, 3);
		\draw[->, red, very thick] (3.75, 0) -- (3.75, 4);
		\draw[->, red, very thick] (4.5, 0) -- (4.5, 5);
		\draw[->, red, very thick] (5.25, 0) -- (5.25, 6);
		\draw[->, red, very thick] (6, 0) -- (6, 5);
		\draw[->, red, very thick] (6.75, 0) -- (6.75, 4);
		\draw[->, red, very thick] (7.5, 0) -- (7.5, 3);
		\draw[->, red, very thick] (8.25, 0) -- (8.25, 2);
		\draw[->, red, very thick] (9, 0) -- (9, 1);

		\draw node at (1.5, 1.3) {$\frac{1}{36}$};
		\draw node at (2.25, 2.3) {$\frac{2}{36}$};
		\draw node at (3, 3.3) {$\frac{3}{36}$};
		\draw node at (3.75, 4.3) {$\frac{4}{36}$};
		\draw node at (4.5, 5.3) {$\frac{5}{36}$};
		\draw node at (5.25, 6.3) {$\frac{6}{36}$};
		\draw node at (6, 5.3) {$\frac{5}{36}$};
		\draw node at (6.75, 4.3) {$\frac{4}{36}$};
		\draw node at (7.5, 3.3) {$\frac{3}{36}$};
		\draw node at (8.25, 2.3) {$\frac{2}{36}$};
		\draw node at (9, 1.3) {$\frac{1}{36}$};

	\end{tikzpicture}
\end{center}


\textbf{e.}
Calcule a probabilidade de obter um valor no intervalo $[7, 9]$.

Deveremos calcular a probalilidade para que $7 \leq X \leq 9$.

\[P_x(7 \leq X \leq 9) = F_x(9) - \lim_{\epsilon\to0} F_x(7 - \epsilon)\]
\[P_x(7 \leq X \leq 9) = \frac{30}{36} - \frac{15}{36} = \frac{15}{36}\]
\[P_x(7 \leq X \leq 9) = \frac{5}{12}\]

% -------------------------------------------------------------------------------

\section*{Questão 2}
Seja x uma variável aleatória discreta com função de probabilidade dada por

\[P(x=X) = \frac{c}{4^x}, X = 0, 1, \cdots\]

Obtenha:

\textbf{a.} O valor de $c$.

Para que isso possa expressar uma probabilidade, deve cumprir o fato de que a soma de todos os casos vale 1. Assim, façamos a soma para infinitos termos.
\[\sum_{x = 0}^{\infty} \frac{c}{4^x} = 1\]
\[c \sum_{x = 0}^{\infty} \frac{1}{4^x} = 1\]

Sabe-se que o somatório acima é a soma de infinitos termos de uma P.G. infinita, que é dada por $\frac{a_1}{1 - q}$ em que $a_1$ é o primeiro termo e $q$ a razão. Então
\[c \left(\frac{1}{1 - 1/4}\right) = 1\]
\[c = \frac{3}{4}\]

\textbf{b.} A probabilidade de $x$ ser um número par.

Sabemos que o valor de $c$ é $\frac{3}{4}$. Queremos agora que $X$ assuma os valores $0, 2, 4, 6, \cdots$ tendo uma outra P.G. infinita. Assim:

\[P(x_{par}) = \frac{3}{4} \left(\frac{1}{4^0} + \frac{1}{4^2} + \cdots \right)\]
sendo, novamente, uma P.G. infinita, com razão $\frac{1}{16}$ e primeiro termo $1$. Dessa forma:
\[P(x_{par}) = \frac{3}{4} \cdot \frac{1}{1 - 1/16} = \frac{3}{4} \cdot \frac{16}{15} = \frac{4}{5}\]

% -------------------------------------------------------------------------------

\section*{Questão 3}
Seja $x$ uma variável aleatória com função densidade de probabilidade

\begin{equation*}
	p_x(X) =
	\begin{cases}
		c\left(1 - x^2 \right) ; & -1 < x < 1 \\
		0 \hspace{1.4cm}; & \text{caso contrário}
	\end{cases}       
\end{equation*}

\textbf{a.} Qual o valor de $c$?

Aqui, sabemos que a integral de $p_x(X)$ é igual a 1. Portanto:
\[c\int_{-1}^{1} p_x(X)\,dx = 1\]
\[c\int_{-1}^{1} (1 - x^2)\,dx = 1\]
\[c \cdot \left(x - \frac{x^3}{3}\right) \bigg|_{-1}^{1} = 1\]
\[c \cdot \frac{4}{3} = 1\]
\[c = \frac{3}{4}\]

\textbf{b.} qual é a Função Distribuição Cumulativa de $x$?

Sabemos que
\[F_x(X) = \int_{-\infty}^{X} p_x(X)\,dx\]

que, para efeitos desta questão, é o mesmo que
\[F_x(X) = \int_{-1}^{X} \frac{3}{4} \cdot \left(1 - x^2\right)\,dx\]

Sabendo que, para isso, $X \leq 1$.
\[F_x(X) = \frac{3}{4} \cdot \left(x - \frac{x^3}{3}\right) \bigg|_{-1}^{X}\]
\[F_x(X) = \frac{3}{4} \cdot \left(-\frac{X^3}{3} + X + \frac{2}{3}\right)\]
\[F_x(X) = -\frac{X^3}{4} + \frac{3X}{4} + \frac{1}{2}\]

% -------------------------------------------------------------------------------

\section*{Questão 4}
Quinze pessoas portadoras de determinada doença são selecionadas para se submeter
a um tratamento. Este tratamento é eficaz na cura da doença em 80\% dos casos.
Suponha que os indivíduos submetidos ao tratamento se curam (ou não) independentes
uns do outros.

\textbf{a.} Modelar com uma v.a. o número de curados x dentre os 15 pacientes submetidos ao tratamento.
\[\Omega_x = \left\{0, 1, 2, \cdots, 15\right\}\]

Os sucessos serão os casos descritos em $\Omega_x$.

\textbf{b.}
Qual a distribuição de $x$?

Queremos obter a probabilidade de $x = K$ pacientes alcançarem sucesso - que seria a cura. Podemos modelar da seguinte forma:
\[P(x = K) = \binom{n}{K} p^K (1 - p)^{n - K}\]
Aqui, escolhemos $K$ pacientes dentre 15. Fazemos o produto dos $p's$ $K$ vezes - escolhendo $K$ que obtiveram sucesso, ou seja, $1 - K$ que não obtiveram sucesso. Por fim, teremos:
\[P(x = K) = \binom{15}{K} p^K (1 - p)^{15 - K}\]

\textbf{c.}
Qual a probabilidade de que os 15 pacientes sejam curados?

Da fórmula anterior, obtemos:
\[P(x = 15) = \binom{15}{15} 0.8^{15} \cdot (1 - 0.8)^{15 - 15}\]
\[P(x = 15) = 0.8^15 = 0.03518\]
\[P(x = 15) \approx 3.52\%\]


\textbf{d.}
Qual a probabilidade de que pelo menos dois não sejam curados?

Aqui, queremos $P(x \leq 13)$.

Porém, sabemos que
\[P(x \leq 13) = 1 - P(x > 13) = 1 - P(x = 14) - P(x = 15)\]
\[P(x \leq 13) = 1 - (1 \cdot (0.8)^{14} \cdot (0.2)^{1}) - 0.03518\]
\[P(x \leq 13) = 1 - 0.00879 - 0.03518 = 0.956\]
\[P(x \leq 13) \approx 95.6\%\]

% -------------------------------------------------------------------------------

\section*{Questão 5}
Seja $x$ o número de ensaios necessários para obter o primeiro sucesso quando se realiza uma sequência de ensaios de Bernoulli independentes com probabilidade de sucesso $p$.

\textbf{a.}
Deduza o modelo de distribuição de probabilidades conhecida como geométrica.

Aqui, teremos um modelo similar ao da questão anterior, porém, não iremos escolher quando haverá sucesso. Dessa forma, teremos
\[P(x) = (1 - p)^{x - 1}p\]

Sendo que há fracasso em $x - 1$ vezes e sucesso apenas uma vez, sendo a última.

\textbf{b.}
Mostre que a somatória das probabilidades da distribuição geométrica é igual a 1.

Vamos, então, calcular a somatória para infinitos valores de $X$. Queremos, então:
\[\sum_{X = 1}^{\infty} (1 - p)^{x - 1}p = p \cdot \sum_{X = 1}^{\infty} (1 - p)^{x - 1}\]

Caímos, aqui, em uma P.G. infinita, com primeiro termo 1 e razão $1 - p$.
\[p \cdot \sum_{X = 1}^{\infty} (1 - p)^{x - 1} = p \cdot \frac{1}{1 - (1 - p)} = \frac{p}{p} = 1\]

% -------------------------------------------------------------------------------

\section*{Questão 6}
Um amperímetro digital tem uma escala que vai de -5 a +5 Ampère e tem a precisão de apenas um dígito, isto é, indica valores inteiros de corrente. Assim, ao se fazer uma medida, o aparelho aproxima o valor real da corrente para o valor inteiro mais próximo. Determine a probabilidade de que o erro na medida seja superior a $0.2$ Ampère. Considere que a função densidade de probabilidade da
corrente na entrada do amperímetro seja a mostrada na figura 4.

\begin{figure}[ht]
	\begin{center}
		\begin{tikzpicture}
			\draw[->] (-3, 0) -- (4, 0);
			\draw[->] (0, 0) -- (0, 3.5);
			\draw (-2.5, 0) -- (0, 2.5);
			\draw (0, 2.5) -- (2.5, 0);
			\draw (-2.5, -0.2) -- (-2.5, 0);
			\draw (0, -0.2) -- (0, 0);
			\draw (2.5, -0.2) -- (2.5, 0);
			\draw (1.6, -0.2) -- (1.6, 0);
			\draw node at (-2.5, -0.5) {-5,0};
			\draw node at (0, -0.5) {0,0};
			\draw node at (2.5, -0.5) {5,0};
			\draw node at (1.6, -0.5) {3,0};
			\draw node at (4, -0.5) {$I$};
			\draw node at (0.5, 3.5) {$p_i(I)$};
			\draw[->, very thick] (1.6, 0) -- (1.6, 2);
			\draw node at (2.1, 2) {$(0, 2)$};
		\end{tikzpicture}

		Figura 4
	\end{center}
\end{figure}

É preciso, primeiramente, refazermos a figura acima, sabendo o valor de $p_i(0)$, que não é mostrado. Assim, sabendo que

\[\int_{-\infty}^{\infty} p_i(I)\,di = 1\]

ou, mais convenientemente, que a área do triângulo somada a integral do impulso de amplitude $0.2$ deve ser 1.

\[\frac{10 c}{2} + 0.2 = 1\]
\[c = 0.16\]

Então, recriando a figura, teremos

\begin{figure}[ht]
	\begin{center}
		\begin{tikzpicture}
			\draw[->] (-3, 0) -- (4, 0);
			\draw[->] (0, 0) -- (0, 3.5);
			\draw (-2.5, 0) -- (0, 2.5);
			\draw (0, 2.5) -- (2.5, 0);
			\draw (-2.5, -0.2) -- (-2.5, 0);
			\draw (0, -0.2) -- (0, 0);
			\draw (2.5, -0.2) -- (2.5, 0);
			\draw (1.6, -0.2) -- (1.6, 0);
			\draw node at (-2.5, -0.5) {-5,0};
			\draw node at (0, -0.5) {0,0};
			\draw node at (2.5, -0.5) {5,0};
			\draw node at (1.6, -0.5) {3,0};
			\draw node at (4, -0.5) {$I$};
			\draw node at (0.5, 3.5) {$p_i(I)$};
			\draw[->, very thick] (1.6, 0) -- (1.6, 2);
			\draw node at (2.1, 2) {$(0, 2)$};
			\draw node at (0.5, 2.5) {$0.16$};
		\end{tikzpicture}
	\end{center}
\end{figure}

Agora, temos uma função clara para $p_i(I)$ que é dada por:
\[p_i(I) = 0.16 \cdot \left(\frac{I}{5} + 1\right) \cdot u_{-1}(I + 5) -
\frac{0.32I}{5}\cdot u_{-1}(I) +
0.2\delta(I - 3)
\]

Dessa forma, podemos calcular a Função Distribuição de Probabilidades, pela integral da equação acima
\[F_i(I) = \int_{-\infty}^{I} p_i(I)\,dI\]
\[F_i(I) = 0.16 \cdot \left(\frac{I^2}{10} + I\right) \bigg|_{-5}^{I} \cdot u_{-1}(I + 5) -
\frac{0.32I^2}{10} \bigg|_{-5}^{I} \cdot u_{-1}(I) +
0.2 \cdot u_{-1}(I - 3)
\]
\[F_i(I) = 0.16 \cdot \left(\frac{I^2}{10} + I + 2.5\right) \cdot u_{-1}(I + 5) -
\left(\frac{0.32I^2}{10} - 0.8\right) \cdot u_{-1}(I) +
0.2 \cdot u_{-1}(I - 3)
\]

% -------------------------------------------------------------------------------

\section*{Questão 7}
No canal binário da figura, a v.a. $x_1$ representa os dígitos transmitidos e a v.a. $x_2$ os dígitos recebidos. Os bits recebidos podem estar alterados devido ao ruído no canal.

\begin{figure}[ht]
	\begin{center}
		\begin{tikzpicture}
			\filldraw (0, 0) circle (0.05);
			\filldraw (4, 0) circle (0.05);
			\filldraw (4, 2) circle (0.05);
			\filldraw (0, 2) circle (0.05);
			\draw (0, 0) -- (4, 0);
			\draw (4, 0) -- (0, 2);
			\draw (0, 2) -- (4, 2);
			\draw (4, 2) -- (0, 0);

			\draw node at (-0.2, -0.2) {$1$}; % 1 (esquerda)
			\draw node at (4.2, -0.2) {$1$}; % 1 (direita)
			\draw node at (-0.2, 2.2) {$0$}; % 0 (esquerda)
			\draw node at (4.2, 2.2) {$0$}; % 0 (direita)

			\draw node at (2, 1.8) {$p$}; %p (cima)
			\draw node at (2, 0.2) {$p$}; %p (baixo)

			\draw node at (0.5, 0.6) {$1 - p$}; % 1 - p (baixo)
			\draw node at (0.5, 1.4) {$1 - p$}; % 1 - p (cima)

			\draw node at (-0.5, 1) {$x_1$};
			\draw node at (4.5, 1) {$x_2$};

		\end{tikzpicture}
	\end{center}
\end{figure}

\textbf{a.} Determine os eventos Ax para um vetor aleatório bidimensional que represente este experimento.

\textbf{b.} Encontre a distribuição de probabilidades e a F.D.P conjuntas, considerando que a probabilidade de transmissão de cada um dos dígitos é igual, e que o ruído do canal pode ser caracterizado pelas probabilidades condicionadas mostradas na figura.

\textbf{c.} Encontre a distribuição de probabilidades e a F.D.P de cada v.a.

% -------------------------------------------------------------------------------

\section*{Questão 8}
Encontre a distribuição de probabilidades marginal $P_x(K)$, sabendo que:

\begin{equation*}
	P_{xy}(K, L) = 
	\begin{cases}
		\frac{2\left[\frac{K}{K + 1}\right]^L}{n(n + 1)} & K = 0, 1, \cdots, n - 1; L \geq 0\\
		0 \hspace{1cm}                                   & \text{Caso contrário} 
	\end{cases}
\end{equation*}

Observação

\[
	P_x(X_i) = 	\sum_{j} P_{xy}(X_i, Y_j), \hspace{0.5cm}
	P_y(Y_j) =  \sum_{i} P_{xy}(X_i, Y_j)
\]

% -------------------------------------------------------------------------------

\section*{Questão 9}
Dois usuários A e B solicitam serviços a um determinado provedor. Considere que as v.a.'s $x$ e $y$ caracterizam, respectivamente, os tempos (em segundos)necessários para a execução dos serviços solicitados pelos usuários A e B.
Suponha que a densidade de probabilidade conjunta das v.a.'s $x$ e $y$ seja dada por:

\[p_{xy}(X, Y) = Ke^{-(3X + 2Y)}u(X)u(Y)\]

Sabe-se que quando um serviço é solicitado ao provedor, a probabilidade de que
a solicitação venha de A vale $0.6$, valendo, portanto $0.4$ a probabilidade de que ela venha de B.

\textbf{a.} Determine o valor da constante $K$.


\textbf{b.} Calcule a probabilidade de que a execução de um serviço solicitado ao provedor requeira um tempo superior a $0.5$ segundos.


\textbf{c.} Calcule a probabilidade de que uma solicitação de serviço venha do usuário A sabendo-se que sua execução requer um tempo maior do que $0.5$ segundos.

% -------------------------------------------------------------------------------

\section*{Questão 10}
Se $x$ é uma variável aleatória normal (gaussiana) com parâmetros $\mu = 10$ e $\sigma^2 = 36$, calcule:

\textbf{a.} $P(x > 5)$

\textbf{b.} $P(4 < x< 16)$

\textbf{c.} $P(x < 8)$

\textbf{d.} $P(x < 20)$

\textbf{e.} $P(x > 16)$
	
\end{document}

