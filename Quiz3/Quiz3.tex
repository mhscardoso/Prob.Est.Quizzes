\documentclass[a5paper]{report}

\usepackage[utf8]{inputenc}
\usepackage[brazilian]{babel}
\usepackage{geometry}
\usepackage{amsmath}
\usepackage{amsfonts}
\usepackage{tikz, pgfplots}
\usepackage{setspace}

\newgeometry{
	margin=0.7in
}

\setstretch{1.25}
\setlength{\parindent}{0pt}
\pgfplotsset{compat=1.18}

\title{Estatística e Modelos Probabilísticos - Quiz 3}
\author{
	Matheus Henrique Sant'Anna Cardoso\\
	DRE: 121073530
}
\date{Novembro de 2022}


\begin{document}
	
\maketitle \newpage

\section*{Questão 1}
Considere o lançamento de dois dados e a experiência cujo resultado consiste na soma do número de pontos da face superior dos dados. Resolva:

\textbf{a.}
Modele com uma v.a. esta soma.


\textbf{b.}
Encontre e esboce sua distribuição de probabilidade.


\textbf{c.}
Encontre e esboce a Função Distribuição de Probabilidades (F.D.P.)


\textbf{d.}
Encontre e esboce a função densidade de probabilidades (f.d.p.)


\textbf{e.}
Calcule a probabilidade de obter um valor no intervalo $[7, 9]$.

% -------------------------------------------------------------------------------

\section*{Questão 2}
Seja x uma variável aleatória discreta com função de probabilidade dada por

\[P(x=X) = \frac{c}{4^x}, X = 0, 1, \cdots\]

Obtenha:

\textbf{a.} O valor de $c$.

\textbf{b.} A probabilidade de $x$ ser um número par.

% -------------------------------------------------------------------------------

\section*{Questão 3}
Seja $x$ uma variável aleatória com função densidade de probabilidade

\begin{equation*}
	p_x(x) =
	\begin{cases}
		c\left(1 - x^2 \right) ; & -1 < x < 1 \\
		0 \hspace{1.4cm}; & \text{caso contrário}
	\end{cases}       
\end{equation*}

\textbf{a.} Qual o valor de $c$?

\textbf{b.} qual é a Função Distribuição Cumulativa de $x$?

% -------------------------------------------------------------------------------

\section*{Questão 4}
Quinze pessoas portadoras de determinada doença são selecionadas para se submeter
a um tratamento. Este tratamento é eficaz na cura da doença em 80\% dos casos.
Suponha que os indivíduos submetidos ao tratamento se curam (ou não) independentes
uns do outros.

\textbf{a.} Modelar com uma v.a. o número de curados x dentre os 15 pacientes submetidos
ao tratamento.


\textbf{b.}
Qual a distribuição de $x$?


\textbf{c.}
Qual a probabilidade de que os 15 pacientes sejam curados?


\textbf{d.}
Qual a probabilidade de que pelo menos dois não sejam curados?

% -------------------------------------------------------------------------------

\section*{Questão 5}
Seja $x$ o número de ensaios necessários para obter o primeiro sucesso quando se realiza
uma sequência de ensaios de Bernoulli independentes com probabilidade de sucesso
$p$.

\textbf{a.} Deduza o modelo de distribuição de probabilidades (conhecida como geométrica).


\textbf{b.} Mostre que a somatória das probabilidades da distribuição geométrica é igual a 1.

% -------------------------------------------------------------------------------

\section*{Questão 6}
Um amperímetro digital tem uma escala que vai de -5 a +5 Ampère e tem a precisão de apenas um dígito, isto é, indica valores inteiros de corrente. Assim, ao se fazer uma medida, o aparelho aproxima o valor real da corrente para o valor inteiro mais próximo. Determine a probabilidade de que o erro na medida seja superior a $0.2$ Ampère. Considere que a função densidade de probabilidade da
corrente na entrada do amperímetro seja a mostrada na figura 4.

\begin{figure}[ht]
	\begin{center}
		\begin{tikzpicture}
			\draw[->] (-3, 0) -- (4, 0);
			\draw[->] (0, 0) -- (0, 3.5);
			\draw (-2.5, 0) -- (0, 2.5);
			\draw (0, 2.5) -- (2.5, 0);
			\draw (-2.5, -0.2) -- (-2.5, 0);
			\draw (0, -0.2) -- (0, 0);
			\draw (2.5, -0.2) -- (2.5, 0);
			\draw (1.6, -0.2) -- (1.6, 0);
			\draw node at (-2.5, -0.5) {-5,0};
			\draw node at (0, -0.5) {0,0};
			\draw node at (2.5, -0.5) {5,0};
			\draw node at (1.6, -0.5) {3,0};
			\draw node at (4, -0.5) {$I$};
			\draw node at (0.5, 3.5) {$p_i(I)$};
			\draw[->, very thick] (1.6, 0) -- (1.6, 2);
			\draw node at (2.1, 2) {$(0, 2)$};
		\end{tikzpicture}

		Figura 4
	\end{center}
\end{figure}

% -------------------------------------------------------------------------------

\section*{Questão 7}
No canal binário da figura, a v.a. $x_1$ representa os dígitos transmitidos e a v.a. $x_2$ os dígitos recebidos. Os bits recebidos podem estar alterados devido ao ruído no canal.

\begin{figure}[ht]
	\begin{center}
		\begin{tikzpicture}
			\filldraw (0, 0) circle (0.05);
			\filldraw (4, 0) circle (0.05);
			\filldraw (4, 2) circle (0.05);
			\filldraw (0, 2) circle (0.05);
			\draw (0, 0) -- (4, 0);
			\draw (4, 0) -- (0, 2);
			\draw (0, 2) -- (4, 2);
			\draw (4, 2) -- (0, 0);

			\draw node at (-0.2, -0.2) {$1$}; % 1 (esquerda)
			\draw node at (4.2, -0.2) {$1$}; % 1 (direita)
			\draw node at (-0.2, 2.2) {$0$}; % 0 (esquerda)
			\draw node at (4.2, 2.2) {$0$}; % 0 (direita)

			\draw node at (2, 1.8) {$p$}; %p (cima)
			\draw node at (2, 0.2) {$p$}; %p (baixo)

			\draw node at (0.5, 0.6) {$1 - p$}; % 1 - p (baixo)
			\draw node at (0.5, 1.4) {$1 - p$}; % 1 - p (cima)

			\draw node at (-0.5, 1) {$x_1$};
			\draw node at (4.5, 1) {$x_2$};

		\end{tikzpicture}
	\end{center}
\end{figure}

\textbf{a.} Determine os eventos Ax para um vetor aleatório bidimensional que represente este experimento.

\textbf{b.} Encontre a distribuição de probabilidades e a F.D.P conjuntas, considerando que a probabilidade de transmissão de cada um dos dígitos é igual, e que o ruído do canal pode ser caracterizado pelas probabilidades condicionadas mostradas na figura.

\textbf{c.} Encontre a distribuição de probabilidades e a F.D.P de cada v.a.

% -------------------------------------------------------------------------------

\section*{Questão 8}
Encontre a distribuição de probabilidades marginal $P_x(K)$, sabendo que:

\begin{equation*}
	P_{xy}(K, L) = 
	\begin{cases}
		\frac{2\left[\frac{K}{K + 1}\right]^L}{n(n + 1)} & K = 0, 1, \cdots, n - 1; L \geq 0\\
		0 \hspace{1cm}                                   & \text{Caso contrário} 
	\end{cases}
\end{equation*}

Observação

\[
	P_x(X_i) = 	\sum_{j} P_{xy}(X_i, Y_j), \hspace{0.5cm}
	P_y(Y_j) =  \sum_{i} P_{xy}(X_i, Y_j)
\]

% -------------------------------------------------------------------------------

\section*{Questão 9}
Dois usuários A e B solicitam serviços a um determinado provedor. Considere que as v.a.'s $x$ e $y$ caracterizam, respectivamente, os tempos (em segundos)necessários para a execução dos serviços solicitados pelos usuários A e B.
Suponha que a densidade de probabilidade conjunta das v.a.'s $x$ e $y$ seja dada por:

\[p_{xy}(X, Y) = Ke^{-(3X + 2Y)}u(X)u(Y)\]

Sabe-se que quando um serviço é solicitado ao provedor, a probabilidade de que
a solicitação venha de A vale $0.6$, valendo, portanto $0.4$ a probabilidade de que ela venha de B.

\textbf{a.} Determine o valor da constante $K$.


\textbf{b.} Calcule a probabilidade de que a execução de um serviço solicitado ao provedor requeira um tempo superior a $0.5$ segundos.


\textbf{c.} Calcule a probabilidade de que uma solicitação de serviço venha do usuário A sabendo-se que sua execução requer um tempo maior do que $0.5$ segundos.

% -------------------------------------------------------------------------------

\section*{Questão 10}
Se $x$ é uma variável aleatória normal (gaussiana) com parâmetros $\mu = 10$ e $\sigma^2 = 36$, calcule:

\textbf{a.} $P(x > 5)$

\textbf{b.} $P(4 < x< 16)$

\textbf{c.} $P(x < 8)$

\textbf{d.} $P(x < 20)$

\textbf{e.} $P(x > 16)$
	
\end{document}

