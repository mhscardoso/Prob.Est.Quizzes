\documentclass[a5paper]{report}

\usepackage[utf8]{inputenc}
\usepackage[brazilian]{babel}
\usepackage{geometry}
\usepackage{amsmath}
\usepackage{amsfonts}
\usepackage{tikz, pgfplots}
\usepackage{setspace}

\newgeometry{
	margin=0.5in
}

\setstretch{1.5}
\setlength{\parindent}{0pt}
\pgfplotsset{compat=1.18}

\title{Estatística e Modelos Probabilísticos - Quiz 4}
\author{
	Matheus Henrique Sant'Anna Cardoso\\
	DRE: 121073530
}
\date{Novembro de 2022}

\begin{document}
    \maketitle \newpage

\section*{Questão 1}
\textbf{Considere duas variáveis aleatórias $x$ e $y$ com função densidade de probabilidade conjunta constante e diferente de zero apenas na área hachurada da figura. Determine o valor de $p_{xy}(X, Y)$ na área hachurada de figura. Encontre a função densidade de probabilidade $p_x(X)$ da variável aleatória $x$. Determine a função densidade de probabilidade condicional $p_{x \mid M}(X)$ onde $M$ é o evento definido por $M = \{0.4 < y < 0.6\}$.}

\begin{center}
    \begin{tikzpicture}
        \draw[->, thick] (-2.5, 0) -- (2.5, 0);
        \draw[->, thick] (0, -0.2) -- (0, 2.5);
        \draw node at (2.5, -0.4) {$X$};
        \draw node at (0.4, 2.5) {$Y$};

        \filldraw[black!40!white, draw=black, opacity=0.2] (-2, 0) -- (0, 2) -- (2, 0);

        \draw node at (-2, -0.4) {$-1$};
        \draw node at (2, -0.4) {$1$};
        \draw node at (0.4, 2) {$1$};
        \draw node at (0, -0.4) {$0$};

        \draw[thick] (-2, 0) -- (-2, -0.2);
        \draw[thick] (2, 0) -- (2, -0.2);
        \draw[thick] (0, 2) -- (0.2, 2);
    \end{tikzpicture}
\end{center}

Sendo a probabilidade conjunta uma constante, chamemos de $k$. Assim:
\[p_{xy}(X, Y) = k\]
Disso, vem:
\begin{align*}
    \int_{-\infty}^{\infty} \int_{-\infty}^{\infty} p_{xy}(X, Y) \, dX dY &= 1 \\
    \int_{-\infty}^{\infty} \int_{-\infty}^{\infty} k \, dX dY &= 1 \\
    \int_{-1}^{0} \int_{0}^{X + 1} k \, dY dX + \int_{0}^{1} \int_{0}^{-X + 1} k \, dY dX &= 1 \\
    \int_{-1}^{0} (X + 1)k \, dX + \int_{0}^{1} (-X + 1) k \, dX &= 1 \\
    k \left[\frac{X^2}{2} + X\right]_{-1}^{0} + k \left[\frac{-X^2}{2} + X\right]_{0}^{1} &= 1 \\
\end{align*}
\begin{align*}    
    k \left[\frac{1}{2} + \frac{1}{2}\right] &= 1 \\
    k = 1\\
    \boxed{p_{xy}(X, Y) = 1}
\end{align*}

Sigamos para o cálculo de $p_x(X)$ da área hachurada.
\begin{align*}
    p_x(X) = \int_{-\infty}^{\infty} p_{xy}(X, Y) \, dY \\
    \boxed{
    p_x(X) =
    \begin{cases}
        \int_{0}^{X + 1} 1 \, dY = X + 1 & -1 \leq X \leq 0 \\
        \int_{0}^{-X + 1} 1 \, dY = -X + 1 & 0 < X \leq 1
    \end{cases}
    }
\end{align*}

Agora, calculemos $p_{x \mid M}(X)$.

Antes, vejamos $p_y(Y)$
\begin{align*}
    p_y(Y) &= \int_{-\infty}^{\infty} p_{xy}(X, Y) \, dX \\
    p_y(Y) &= \int_{Y - 1}^{0} 1 \, dX + \int_{0}^{-Y + 1} 1 \, dX \,\,\,\,\,\,\, 0 \leq Y \leq 1 \\
    p_y(Y) &= (-Y + 1) + (-Y + 1) \\
    p_y(Y) &= 2(-Y + 1) \,\,\,\,\,\,\, 0 \leq Y \leq 1
\end{align*}

Dessa forma, podemos calcular $P(M)$.
\begin{align*}
    P(0.4 < Y < 0.6) &= 2\int_{0.4}^{0.6} (-Y + 1) \, dY \\
    P(0.4 < Y < 0.6) &= 2 \left[-\frac{Y^2}{2} + Y\right]_{0.4}^{0.6} \\
    P(0.4 < Y < 0.6) &= 2 \left[-\frac{0.6^2}{2} + 0.6 - \left(-\frac{0.4^2}{2} + 0.4\right)\right] \\
    P(0.4 < Y < 0.6) &= 2 \left[0.1\right]
\end{align*}
\begin{align*}
    P(0.4 < Y < 0.6) = 0.2
\end{align*}

Sigamos, pois para o cálculo de $P(x \leq X, M)$. Perceba, que, aqui queremos a interseção, sendo representado pela figura abaixo.
\begin{center}
    \begin{tikzpicture}
        \draw[->, thick] (-2.5, 0) -- (2.5, 0);
        \draw[->, thick] (0, -0.2) -- (0, 2.5);
        \draw node at (2.5, -0.4) {$X$};
        \draw node at (0.4, 2.5) {$Y$};

        \draw (-2, 0) -- (0, 2) -- (2, 0);

        \draw node at (-2, -0.4) {$-1$};
        \draw node at (2, -0.4) {$1$};
        \draw node at (0.4, 2) {$1$};
        \draw node at (0, -0.4) {$0$};

        \draw (-1.5, 1.2) -- (1.5, 1.2);
        \draw (-1.5, 0.8) -- (1.5, 0.8);

        \draw node at (1.9, 1.2) {$0.6$};
        \draw node at (1.9, 0.8) {$0.4$};

        \draw node at (-1.2, -0.4) {\tiny{$-0.6$}};
        \draw node at (-0.8, -0.22) {\tiny{$-0.4$}};
        \draw node at (0.8, -0.4) {\tiny{$0.4$}};
        \draw node at (1.2, -0.4) {\tiny{$0.6$}};

        \filldraw[red!40!white, draw=red, opacity=0.6] (-1.2, 0.8) -- (-0.8, 0.8) -- (-0.8, 1.2);
        \filldraw[green!40!white, draw=green, opacity=0.6] (-0.8, 0.8) -- (-0.8, 1.2) -- (0.8, 1.2) -- (0.8, 0.8);
        \filldraw[violet!40!white, draw=violet, opacity=0.6] (1.2, 0.8) -- (0.8, 0.8) -- (0.8, 1.2);

        \draw[dashed] (-1.2, 0.8) -- (-1.2, -0.2);
        \draw[dashed] (-0.8, 0.8) -- (-0.8, -0.2);
        \draw[dashed] (0.8, 0.8) -- (0.8, -0.2);
        \draw[dashed] (1.2, 0.8) -- (1.2, -0.2);

        \draw[thick] (-2, 0) -- (-2, -0.2);
        \draw[thick] (2, 0) -- (2, -0.2);
        \draw[thick] (0, 2) -- (0.2, 2);
    \end{tikzpicture}
\end{center}

Vamos então, analisar por região destacada.
\begin{align*}
    P(x \leq X, M) =
    \begin{cases}
        0 & X < -0.6\\
        \frac{(X + 0.6)^2}{2} & -0.6 \leq X \leq -0.4\\
        \frac{0.2^2}{2} + (X + 0.4) \cdot 0.2 = 0.1 + 0.2X & -0.4 < X \leq 0.4\\
        0.18 + \frac{(0.8 - X) \cdot (X - 0.4)}{2} = 0.2 - \frac{(X - 0.6)^2}{2} & 0.4 < X \leq 0.6\\
        0.2 & X > 0.6
    \end{cases}
\end{align*}

Sabemos que
\[p_{x \mid M}(X) = \frac{\frac{d}{dx}P(x \leq X)}{P(M)}\]
Calculando $\frac{d}{dX}P(x \leq X, M)$, chegamos em
\begin{align*}
    \frac{d}{dX}P(x \leq X, M) =
    \begin{cases}
        0 & X < -0.6\\
        X + 0.6 & -0.6 \leq X \leq -0.4\\
        0.2 & -0.4 < X \leq 0.4\\
        0.6 - X & 0.4 < X \leq 0.6\\
        0 & X > 0.6
    \end{cases}
\end{align*}\

Finalmente
\begin{align*}
    \boxed{
    p_{x \mid M}(X) =
    \begin{cases}
        0 & X < -0.6\\
        5X + 3 & -0.6 \leq X \leq -0.4\\
        1 & -0.4 < X \leq 0.4\\
        3 - 5X & 0.4 < X \leq 0.6\\
        0 & X > 0.6
    \end{cases}
    }
\end{align*}

\section*{Questão 2}
\textbf{Uma régua de comprimento unitário é quebrada em um ponto\\ aleatório. O pedaço da esquerda é novamente quebrado. Seja $x$ a variável aleatória que define, a partir da extremidade esquerda da régua, o ponto em que a régua é quebrada pela primeira vez e $y$ a variável aleatória que define o segundo ponto de quebra. Determine}
\[p_x(X), p_y(Y), p_{y \mid x = X}(Y), p_{x \mid y = Y}(X)\]
\textbf{Calcule a probabilidade de que um triangulo possa ser formado com as três peças obtidas. Lembre-se de que, num triangulo, o comprimento de qualquer um dos lados é menor que a soma dos comprimentos dos outros dois lados.}

O cálculo de $p_x(X)$ é bem simples. Tendo uma régua unitária, sua densidade de probabilidade é:
\begin{center}
    \begin{tikzpicture}
        \draw[->, thick] (0, -0.4) -- (0, 1.5);
        \draw[->, thick] (-0.4, 0) -- (1.5, 0);

        \draw node at (-0.2, -0.2) {$0$};
        \draw node at (1, -0.2) {$1$};
        \draw node at (1.5, -0.4) {$X$};
        \draw node at (0.7, 1.5) {$p_x(X)$};
        \draw node at (-0.2, 1) {$1$};

        \draw (0, 1) -- (1, 1) -- (1, 0);
    \end{tikzpicture}
\end{center}
\[p_x(X) = 1; 0 \geq X \geq 1\]

Podemos pensar, agora na probabilidade conjunta, criando um gráfico com Y em função de X.


\section*{Questão 3}

\textbf{Um equipamento pode se encontrar em um dentre dois estados\\ possíveis: operação normal e operação anormal.
A probabilidade de o equipamento encontra-se em operação normal é $0.8$. Ligado a este equipamento tem-se um painel de controle onde um termômetro com escala em graus Celsius indica a temperatura do equipamento a cada instante. Considere que a indicação do termômetro é uma variável aleatória $t$ cuja função densidade de probabilidade depende do estado em que o equipamento se encontra. Se o equipamento está em operação normal $(N)$, tem-se}
\[p_{t \mid N}(T) = \frac{1}{10\sqrt{2\pi}} e^{-\frac{(T - 30)^2}{200}}\]

\textbf{Por outro lado, se o equipamento está em operação anormal $(\overline{N})$, tem-se}
\[p_{t \mid \overline{N}}(T) = \frac{1}{10\sqrt{2\pi}} e^{-\frac{(T - 50)^2}{200}}\]

\textbf{Determine a probabilidade de que a indicação do termômetro exceda 40 graus Celsius.
Determine a probabilidade de que a operação seja normal dado que a indicação do
termômetro está compreendida entre 40 e 50 graus Celsius. 
Suponha que se resolve decidir a respeito do estado do equipamento (operação normal ou anormal) a partir da
indicação do termômetro. Estabelece-se então a seguinte regra de decisão:\\
Se $t < T_0$ decide-se por operação normal\\
Se $t \geq T_0$ decide-se por operação anormal.\\
Calcule a probabilidade de se cometer um erro quando se usa este critério de decisão com
$T_0 = 45$. Calcule o menor valor de $T_0$ tal que a probabilidade de se decidir por operação
anormal quando a operação é normal não excede $0.0075$.}

\section*{Questão 4}
\textbf{Deseja-se encontrar a função de densidade de probabilidade da v.a. $y = ax^2, (a > 0)$, onde $x$ é uma v.a. dupla exponencial de parâmetro $b$:}

\[p_x(X) = \frac{b}{2e^{-b|X|}}\]

\end{document}