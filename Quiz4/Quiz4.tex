\documentclass[a5paper]{report}

\usepackage[utf8]{inputenc}
\usepackage[brazilian]{babel}
\usepackage{geometry}
\usepackage{amsmath}
\usepackage{amsfonts}
\usepackage{tikz, pgfplots}
\usepackage{setspace}

\newgeometry{
	margin=0.5in
}

\setstretch{1.5}
\setlength{\parindent}{0pt}
\pgfplotsset{compat=1.18}

\title{Estatística e Modelos Probabilísticos - Quiz 4}
\author{
	Matheus Henrique Sant'Anna Cardoso\\
	DRE: 121073530
}
\date{Novembro de 2022}

\begin{document}
    \maketitle \newpage

\section*{Questão 1}
Considere duas variáveis aleatórias $x$ e $y$ com função densidade de probabilidade conjunta constante e diferente de zero apenas na área hachurada da figura. Determine o valor de $p_{xy}(X, Y)$ na área hachurada de figura. Encontre a função densidade de probabilidade $p_x(X)$ da variável aleatória $x$. Determine a função densidade de probabilidade condicional $p_{x \mid M}(X)$ onde $M$ é o evento definido por $M = \{0.4 < y < 0.6\}$.

\begin{center}
    \begin{tikzpicture}
        \draw[->, thick] (-2.5, 0) -- (2.5, 0);
        \draw[->, thick] (0, -0.2) -- (0, 2.5);
        \draw node at (2.5, -0.4) {$X$};
        \draw node at (0.4, 2.5) {$Y$};

        \filldraw[black!40!white, draw=black, opacity=0.2] (-2, 0) -- (0, 2) -- (2, 0);

        \draw node at (-2, -0.4) {$-1$};
        \draw node at (2, -0.4) {$1$};
        \draw node at (0.4, 2) {$1$};
        \draw node at (0, -0.4) {$0$};

        \draw[thick] (-2, 0) -- (-2, -0.2);
        \draw[thick] (2, 0) -- (2, -0.2);
        \draw[thick] (0, 2) -- (0.2, 2);
    \end{tikzpicture}
\end{center}

\section*{Questão 2}
Uma régua de comprimento unitário é quebrada em um ponto aleatório. O pedaço da esquerda é novamente quebrado. Seja $x$ a variável aleatória que define, a partir da extremidade esquerda da régua, o ponto em que a régua é quebrada pela primeira vez e $y$ a variável aleatória que define o segundo ponto de quebra. Determine $p_x(X), p_y(Y), p_{y \mid x = X}(Y), p_{x \mid y = Y}(X)$. Calcule a probabilidade de que um triangulo possa ser formado com as três peças obtidas. Lembre-se de que, num triangulo, o comprimento de qualquer um dos lados é menor que a soma dos comprimentos dos outros dois lados.


\section*{Questão 3}
Um equipamento pode se encontrar em um dentre dois estados possíveis:

operação normal e operação anormal. 

A probabilidade de o equipamento encontra-se em operação normal é $0.8$. Ligado a este equipamento tem-se um painel de controle onde um termômetro com escala em graus Celsius indica a temperatura do equipamento a cada instante. Considere que a indicação do termômetro é uma variável aleatória $t$ cuja função densidade de probabilidade depende do estado em que o equipamento se encontra. Se o equipamento está em operação normal $(N)$, tem-se

\[p_{t \mid N}(T) = \frac{1}{10\sqrt{2\pi}} e^{-\frac{(T - 30)^2}{200}}\]

Por outro lado, se o equipamento está em operação anormal $(\overline{N})$, tem-se

\[p_{t \mid \overline{N}}(T) = \frac{1}{10\sqrt{2\pi}} e^{-\frac{(T - 50)^2}{200}}\]

Determine a probabilidade de que a indicação do termômetro exceda 40 graus Celsius.
Determine a probabilidade de que a operação seja normal dado que a indicação do
termômetro está compreendida entre 40 e 50 graus Celsius. 
Suponha que se resolve decidir a respeito do estado do equipamento (operação normal ou anormal) a partir da
indicação do termômetro. Estabelece-se então a seguinte regra de decisão:\\
Se $t < T_0$ decide-se por operação normal\\
Se $t \geq T_0$ decide-se por operação anormal.\\
Calcule a probabilidade de se cometer um erro quando se usa este critério de decisão com
$T_0 = 45$. Calcule o menor valor de $T_0$ tal que a probabilidade de se decidir por operação
anormal quando a operação é normal não excede $0.0075$.

\section*{Questão 4}
Deseja-se encontrar a função de densidade de probabilidade da v.a. $y = ax^2, (a > 0)$, onde $x$ é uma v.a. dupla exponencial de parâmetro $b$:

\[p_x(X) = \frac{b}{2e^{-b|X|}}\]

\end{document}